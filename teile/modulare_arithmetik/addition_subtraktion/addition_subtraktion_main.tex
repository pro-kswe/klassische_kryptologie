% !TEX root = ../../../main.tex

\toggletrue{image}
\toggletrue{imagehover}
\chapterimage{2}
\chapterimagetitle{\uppercase{2}}
\chapterimageurl{https://xkcd.com/2614/}
\chapterimagehover{It's like sigma summation notation, except instead of summing the argument over all values of i, you 2 the argument over all values of 2.}

\chapter{Addition und Subtraktion}
\label{chapter-modulare-addition-und-subtraktion}

Die Grundrechenarten Addition und Subtraktion können wir fast wie gewöhnlich durchführen. Wir müssen einfach immer mit $\bmod$ rechnen. Die Lernziele lauten:

\newcommand{\modulareAdditionSubtraktionLernziele}{
\protect\begin{todolist}
\item Sie verwenden die formale Notation für die modulare Rechnung.
\item Sie führen die modulare Addition und Subtraktion durch.
\end{todolist}
}

\lernziel{\autoref{chapter-modulare-addition-und-subtraktion}, \nameref{chapter-modulare-addition-und-subtraktion}}{\protect\modulareAdditionSubtraktionLernziele}

\modulareAdditionSubtraktionLernziele

\vspace{-0.25cm}

\section{Modulare Addition}

Wir addieren zwei natürliche Zahlen wie gewohnt und berücksichtigen dann den Modulo-Operator.

\begin{definition}[Modulare Addition]
Wir addieren zwei natürliche Zahlen $x$ und $y$ und berechnen anschliessend den Rest der ganzzahligen Division mit $a$. Wir notieren dies wie folgt:

\begin{center}
$x \oplus_a y = (x + y) \bmod a$
\end{center}

Wir nennen diese Rechnung Addition Modulo $a$.

\end{definition}

\begin{example}
Folgende Zahlen sind gegeben: $x = 15$, $y = 17$ und $a = 26$. Wir berechnen somit $15 \oplus_{26} 17 = (15 + 17) \bmod 26 = 32 \bmod 26 = 6$.
\end{example}

\subsection{Aufgaben}

\begin{enumerate}
	\item Schreiben Sie das Ergebnis der folgenden Berechnungen direkt hinter die Teilaufgabe.

\begin{multicols}{3}
\begin{enumerate}
\item $25 \oplus_{26} 25 = \rule[-0.75mm]{1cm}{.5pt}$
\item $0 \oplus_{26} 19 = \rule[-0.75mm]{1cm}{.5pt}$
\item $231 \oplus_{3} 222 = \rule[-0.75mm]{1cm}{.5pt}$
\item $\num{13874} \oplus_{2} 123 = \rule[-0.75mm]{1cm}{.5pt}$
\item $7 \oplus_{7} 21 = \rule[-0.75mm]{1cm}{.5pt}$
\item $25 \oplus_{8} 13 = \rule[-0.75mm]{1cm}{.5pt}$
\item $9 \oplus_{15} 12 = \rule[-0.75mm]{1cm}{.5pt}$
\item $13 \oplus_{12} 54 = \rule[-0.75mm]{1cm}{.5pt}$
\item $\num{1378} \oplus_{10} \num{24795} = \rule[-0.75mm]{0.7cm}{.5pt}$
\end{enumerate}
\end{multicols}

\item Beantworten Sie die folgenden Fragen mit einer passenden modularen Addition.

\begin{enumerate}
	\item Es ist 09:00 Uhr. Wie spät ist es nach \num{100} Stunden?
	
	\fillwithgrid{0.5in}
	
	\item Heute ist Montag/Dienstag/Mittwoch/Donnerstag/Freitag/Samstag/Sonntag. Welchen Wochentag haben wir in \num{53} Tagen?
	
	\fillwithgrid{0.5in}
\end{enumerate}

\end{enumerate}

\section{Modulare Subtraktion}

Bei der Subtraktion kann es vorkommen, dass wir eine negative Zahl erhalten. In der modularen Arithmetik werden in der Regel \textbf{keine negativen Zahlen} benutzt. Wenn wir mit $\bmod a$ rechnen, dann erhalten wir immer natürliche Zahlen aus dem Bereich $0, 1, 2, \dots, a-2, a-1$. Dies gilt auch für die modulare Addition. Wir müssen die Subtraktion also so definieren, dass beim Subtrahieren nur Zahlen aus $\mathbb{Z}_a$ entstehen.

\subsection{Kochrezept für die modulare Subtraktion}

Wir führen die modulare Subtraktion $x \ominus_a y$ wie folgt durch: 

\begin{enumerate}
\item Subtrahiere zunächst $x$ und $y$ ohne modulare Berechnungen: $z = x - y$.
\item Falls $z$ \textbf{positiv} ist, dann berechne $z \bmod a$. Dies ist dann das Ergebnis von $x \ominus_a y$.
\item Sonst ($z$ ist \textbf{negativ}) addiere so oft $a$ zu $z$, bis die Zahl erstmalig \textbf{nicht negativ} ist.
\end{enumerate}

Wir berechnen nun drei Beispiele.

\begin{example}
Sei $x = 21$, $y = 10$ und $a = 5$. Wir berechnen $21 \ominus_{5} 10$ wie folgt: 
\begin{itemize}
\item $z = x - y = 21 - 10 = 11$
\item Da $z = 11$ positiv ist, können wir direkt $z \bmod a$ berechnen. Wir erhalten $11 \bmod 5 = 1$.
\end{itemize}
Somit gilt: $21 \ominus_{5} 10 = 1$.
\end{example}

\begin{example}
Sei $x = 15$, $y = 17$ und $a = 26$. Wir berechnen $15 \ominus_{26} 17$ wie folgt:
\begin{itemize}
\item $z = x - y = 15 - 17 = -2$
\item Da $z = -2$ negativ ist, müssen wir $a = 26$ zu $z = -2$ addieren. Wir erhalten $z + a = -2 + 26 = 24$. Die Zahl ist nicht negativ und wir haben das korrekte Ergebnis berechnet.
\end{itemize}
Somit gilt: $15 \ominus_{5} 17 = 24$.
\end{example}

\begin{example}
Sei $x = 3$, $y = 50$ und $a = 26$. Wir berechnen $3 \ominus_{26} 50$ wie folgt: 
\begin{itemize}
\item $z = x - y = 3 - 50 = -47$
\item Da $z = -47$ negativ ist, müssen wir $a = 26$ zu $z = -47$ addieren. Wir erhalten $z + a = -47 + 26 = -21$. Die Zahl ist noch immer negativ.
\item Wir addieren erneut $a = 26$ und erhalten $-21 + 26 = 5$. Die Zahl ist nicht negativ und wir haben das korrekte Ergebnis berechnet.
\end{itemize}
Somit gilt: $3 \ominus_{26} 50 = 5$.
\end{example}

\subsection{Aufgaben}

Schreiben Sie das Ergebnis der folgenden Berechnungen direkt hinter die Teilaufgabe. Sie können unten Nebenrechnungen durchführen.

\begin{multicols}{2}
\begin{enumerate}
\item $25 \ominus_{26} 25 = \rule[-0.75mm]{2cm}{.5pt}$
\item $0 \ominus_{26} 19 = \rule[-0.75mm]{2cm}{.5pt}$
\item $231 \ominus_{3} 222 = \rule[-0.75mm]{2cm}{.5pt}$
\item $\num{1378} \ominus_{10} \num{24795} = \rule[-0.75mm]{2cm}{.5pt}$
\item $\num{13874} \ominus_{2} 123 = \rule[-0.75mm]{2cm}{.5pt}$
\item $7 \ominus_{7} 21 = \rule[-0.75mm]{2cm}{.5pt}$
\end{enumerate}
\end{multicols}

\fillwithgrid	{\stretch{1}}
